\documentclass[final, finnished, monochromatic]{jyuposter}

\addbibresource{thesis.bib}

\newcommand \jyuauthor{Author Name, Other Author Name}
\newcommand \jyuauthorfinnish{Tekijän Nimi, Toisen Tekijän Nimi}
\newcommand \jyutitle{Title of Work}
\newcommand \jyutitlefinnish{Työn otsikko}
\newcommand \jyupublication{Type of Work}
\newcommand \jyupublicationfinnish{Työn tyyppi}
\newcommand \jyusupervisor{Supervisor Name, Other Supervisor Name}
\newcommand \jyusupervisorfinnish{Ohjaajan Nimi, Toisen Ohjaajan Nimi}
\newcommand \jyuinstitute{University of Jyväskylä}
\newcommand \jyuinstitutefinnish{Jyväskylän yliopisto}
\newcommand \jyufaculty{Faculty of Information Technology}
\newcommand \jyufacultyfinnish{Informaatioteknologian tiedekunta}
% \newcommand \jyudepartment{Department of Mathematical Information Technology}
% \newcommand \jyudepartmentfinnish{Tietotekniikan laitos}
\newcommand \jyudepartment{}
\newcommand \jyudepartmentfinnish{}
\newcommand \jyuurl{https://www.jyu.fi/it/en/}
\newcommand \jyuurlfinnish{https://www.jyu.fi/it/fi/}
\newcommand \jyuslogan{Serving the Future since 1863.}
\newcommand \jyusloganfinnish{Tulevaisuuden palveluksesssa jo vuodesta 1863.}
\newcommand \jyushortslogan{Since 1863.}
\newcommand \jyushortsloganfinnish{Vuodesta 1863.}
\newcommand \jyuyear{2017}
\newcommand \jyumonth{05}
\newcommand \jyuday{01}

\makesetup

% These commands are optional.
\newcommand \yesnumber{\addtocounter{equation} 1\tag \theequation}
\newcommand \full{{\mathrm d}}

\begin{document}

\maketitle

\begin{multicols} 3

\noindent
\begin{tcolorbox}
\section*{Abstract}
\fontfamily{fla}\fontseries b\fontshape n\fontsize{18}{27}\selectfont

Yhtälö
\begin{align*}
  \phi (R', R'') & = \int \full x \chi (t, x)^* \psi (t, x)
  \yesnumber \label{e/equation}
\end{align*}
löytyy myöhemmin vielä tekstin seasta numerolla \ref{e/equation-again}.

\lipsum[1]
\end{tcolorbox}

\section*{Piirasta}

\lipsum[1-3]

The logo can be customized parametrically, globally or per invocation.

\tikzset{
  jyutorch/.style={line width=0.2mm},
  jyuhandle/.style={fill=none, draw=jyublue},
  jyuflame/.style={fill=none, draw=jyublue}}

\bigskip
\noindent
% \includegraphics[width=\columnwidth]{pie}
\fbox{\resizebox{150mm}{80mm} \jyulogo}

\tikzset{
  jyuhandle/.style={fill=jyublue, draw=none},
  jyuflame/.style={fill=jyuorange, draw=none}}

\lipsum[4-5]

\bigskip
\noindent
\begin{minipage} \columnwidth
\begin{flushright}
% \includegraphics[width=\columnwidth]{equipment}
\fbox{\resizebox{150mm}{240mm} \jyulogo}
\\
\fontfamily{fla}\fontseries m\fontshape{it}\fontsize{18}{22}\selectfont
Tässä on kaikenlaista kuvatekstiä.
\end{flushright}
\end{minipage}

\lipsum[6-7]

\section*{Kvanttipelleilyä}

Tätä asiakirjaluokkaa voi käyttää vaikka
kvanttimekaniikkaan liittyvän \cite{feynman-1948} yhtälön
\begin{align*}
  \phi (R', R'') & = \int \full x \chi (t, x)^* \psi (t, x)
  \yesnumber \label{e/equation-again}
\end{align*}
kirjoittamiseen.
On tosin parempi olla tekemättä niin.

\bigskip
\noindent
% \includegraphics[width=\columnwidth]{bars}
\fbox{\resizebox{150mm}{120mm} \jyulogo}

\bigskip
\noindent
\begin{tcolorbox}
\section*{Conclusions}

\begin{itemize}
  \item Ergh!
  \item Urgh!
\end{itemize}
\end{tcolorbox}

\nocite{*}

\printbibliography

\end{multicols}

\end{document}
