\documentclass[final, finnished, monochromatic]{jyuposter}

\addbibresource{thesis.bib}

\newcommand \jyuauthor{Author Name, Other Author Name}
\newcommand \jyuauthorfinnish{Tekijän Nimi, Toisen Tekijän Nimi}
\newcommand \jyutitle{Title of Work}
\newcommand \jyutitlefinnish{Työn otsikko}
\newcommand \jyupublication{Type of Work}
\newcommand \jyupublicationfinnish{Työn tyyppi}
\newcommand \jyusupervisor{Supervisor Name, Other Supervisor Name}
\newcommand \jyusupervisorfinnish{Ohjaajan Nimi, Toisen Ohjaajan Nimi}
\newcommand \jyuinstitute{University of Jyväskylä}
\newcommand \jyuinstitutefinnish{Jyväskylän yliopisto}
\newcommand \jyufaculty{Faculty of Information Technology}
\newcommand \jyufacultyfinnish{Informaatioteknologian tiedekunta}
\newcommand \jyudepartment{Department of Mathematical Information Technology}
\newcommand \jyudepartmentfinnish{Tietotekniikan laitos}
\newcommand \jyuurl{https://www.jyu.fi/it/en/}
\newcommand \jyuurlfinnish{https://www.jyu.fi/it/fi/}
\newcommand \jyuslogan{Serving the Future since 1863.}
\newcommand \jyusloganfinnish{Tulevaisuuden palveluksesssa jo vuodesta 1863.}
\newcommand \jyushortslogan{Since 1863.}
\newcommand \jyushortsloganfinnish{Vuodesta 1863.}
\newcommand \jyuyear{2017}
\newcommand \jyumonth{05}
\newcommand \jyuday{01}

\makesetup

% These commands are optional.
\newcommand \yesnumber{\addtocounter{equation} 1\tag \theequation}
\newcommand \full{{\mathrm d}}

\begin{document}

\maketitle

\begin{multicols} 3

\noindent
\begin{tcolorbox}
\section*{Abstract}
\fontfamily{fla}\fontseries b\fontshape n\fontsize{18}{27}\selectfont

Yhtälö
\begin{align*}
  \phi (R', R'') & = \int \full x \chi (t, x)^* \psi (t, x)
  \yesnumber \label{e/equation}
\end{align*}
löytyy myöhemmin vielä tekstin seasta numerolla \ref{e/equation-again}.

\lipsum[1]
\end{tcolorbox}

\section*{Piirasta}

\lipsum[1-2]

Logo purkissa.

\renewcommand \jyuhandlestyle{jyublue}
\renewcommand \jyuflamestyle{jyuorange}
% \fbox{\jyulogo{width=50mm}}
% % This picture was derived from the PostScript source file
% by translating `/mo/moveto`, `/li/lineto`, `/cv/curveto` and `/cp/closepath`.
% We leave `jyuhandle` and `jyuflame` free for customization.
\begin{tikzpicture}[xscale=1, yscale=-1, x=0.1mm, y=0.1mm]
% The source picture is quite peculiar.
% The handle has two parts
% that are specified to a precision of three decimal places.
\filldraw[jyuhandle]
(257.512, 139.681)
.. controls (233.408, 139.681) and (214.211, 125.393) .. (204.656, 110.705)
-- (204.463, 110.409)
-- (310.557, 110.409)
-- (310.373, 110.702)
.. controls (298.939, 128.926) and (279.592, 139.681) .. (257.512, 139.681)
-- cycle;
\filldraw[jyuhandle]
(266.711, 240.861)
-- (248.308, 240.861)
-- (248.308, 144.143)
-- (248.549, 144.210)
.. controls (255.668, 146.157) and (259.369, 146.448) .. (266.465, 144.212)
-- (266.711, 144.133)
-- cycle;
% The flame has four parts,
% but they are specified to a precision of six decimal places.
\filldraw[jyuflame]
(267.338000, 11.798300)
.. controls (276.326000, 20.113200) and (274.309000, 32.512200) .. (272.375000, 36.259700)
.. controls (268.160000, 44.428700) and (258.779000, 54.937000) .. (256.408000, 57.811500)
.. controls (252.453000, 62.606400) and (240.572000, 74.299300) .. (239.287000, 80.762200)
.. controls (238.207000, 86.208000) and (239.916000, 94.199700) .. (240.653000, 97.158600)
.. controls (238.020000, 97.158600) and (233.195000, 97.155700) .. (233.195000, 97.155700)
.. controls (231.297000, 92.242600) and (228.225000, 83.233800) .. (229.989000, 73.049300)
.. controls (231.287000, 65.562400) and (241.873000, 53.722600) .. (243.829000, 51.498000)
.. controls (247.334000, 47.514100) and (257.143000, 35.577600) .. (261.129000, 30.399800)
.. controls (265.672000, 24.501900) and (265.332000, 20.722100) .. (262.277000, 15.980400)
.. controls (262.322000, 16.028300) and (267.338000, 11.798300) .. (267.338000, 11.798300)
-- cycle;
\filldraw[jyuflame]
(259.482000, 107.087000)
-- (246.650000, 107.087000)
.. controls (244.024000, 99.354000) and (240.686000, 89.173800) .. (243.857000, 81.142500)
.. controls (247.268000, 72.506300) and (262.770000, 60.119600) .. (272.395000, 49.342200)
.. controls (277.570000, 42.855400) and (279.510000, 38.771400) .. (276.387000, 33.179100)
.. controls (276.387000, 33.179100) and (281.486000, 28.977500) .. (281.486000, 28.961900)
.. controls (286.928000, 33.636200) and (290.895000, 46.058100) .. (284.967000, 53.727500)
.. controls (281.383000, 58.362700) and (277.672000, 62.325100) .. (272.855000, 67.511700)
.. controls (268.160000, 72.570700) and (263.506000, 76.935000) .. (257.158000, 84.233800)
.. controls (252.301000, 89.818800) and (257.730000, 101.007000) .. (259.482000, 107.087000)
-- cycle;
\filldraw[jyuflame]
(274.195000, 107.087000)
-- (263.305000, 107.087000)
.. controls (262.279000, 103.203000) and (260.643000, 99.126400) .. (260.008000, 95.158600)
.. controls (259.230000, 90.303700) and (259.564000, 87.538500) .. (263.039000, 83.995100)
.. controls (268.016000, 78.922300) and (280.275000, 70.687900) .. (283.219000, 66.920400)
.. controls (285.758000, 63.670800) and (284.252000, 59.470200) .. (284.252000, 59.470200)
.. controls (284.250000, 59.478000) and (289.289000, 55.278800) .. (289.264000, 55.278800)
.. controls (295.498000, 62.517000) and (293.779000, 73.065400) .. (289.107000, 77.040500)
.. controls (285.713000, 79.925700) and (278.418000, 85.955500) .. (273.613000, 90.771900)
.. controls (271.018000, 93.405700) and (270.846000, 93.600000) .. (271.750000, 97.701600)
.. controls (272.502000, 100.847000) and (273.551000, 104.870000) .. (274.195000, 107.087000)
-- cycle;
\filldraw[jyuflame]
(226.463000, 77.370100)
.. controls (223.523000, 73.077100) and (222.094000, 69.014600) .. (221.760000, 63.613700)
.. controls (221.012000, 51.454000) and (229.762000, 41.539000) .. (237.028000, 31.329000)
.. controls (243.256000, 22.577100) and (249.092000, 14.712300) .. (249.795000, 11.304100)
.. controls (250.319000, 8.772900) and (249.273000, 5.995060) .. (248.480000, 4.293400)
.. controls (248.472000, 4.294380) and (253.088000, 0.401314) .. (253.623000, -0.000053)
.. controls (260.328000, 6.125440) and (260.730000, 15.604900) .. (257.334000, 22.763600)
.. controls (249.711000, 38.827100) and (224.246000, 59.649400) .. (226.463000, 77.370100)
-- cycle;
% Without commenting out the end of file,
% the line break at the end would expand the bounding box of the picture.
\end{tikzpicture}%


\bigskip
\noindent
\includegraphics[width=\columnwidth]{pie}

\lipsum[4-5]

\bigskip
\noindent
\begin{minipage} \columnwidth
\begin{flushright}
\includegraphics[width=\columnwidth]{equipment} \\
\fontfamily{fla}\fontseries m\fontshape{it}\fontsize{18}{22}\selectfont
Tässä on kaikenlaista kuvatekstiä.
\end{flushright}
\end{minipage}

\lipsum[6-7]

\section*{Kvanttipelleilyä}

Tätä asiakirjaluokkaa voi käyttää vaikka
kvanttimekaniikkaan liittyvän \cite{feynman-1948} yhtälön
\begin{align*}
  \phi (R', R'') & = \int \full x \chi (t, x)^* \psi (t, x)
  \yesnumber \label{e/equation-again}
\end{align*}
kirjoittamiseen.
On tosin parempi olla tekemättä niin.

\bigskip
\noindent
\includegraphics[width=\columnwidth]{bars}

\bigskip
\noindent
\begin{tcolorbox}
\section*{Conclusions}

\begin{itemize}
  \item Ergh!
  \item Urgh!
\end{itemize}
\end{tcolorbox}

\nocite{*}

\printbibliography

\end{multicols}

\end{document}
