\documentclass[final, finnished, monochromatic]{jyuthesis}

\addbibresource{thesis.bib}

\newcommand \jyuauthor{Author Name}
\newcommand \jyuauthorfinnish{Tekijän Nimi}
\newcommand \jyutitle{Title of Work}
\newcommand \jyutitlefinnish{Työn otsikko}
\newcommand \jyupublication{Type of Work}
\newcommand \jyupublicationfinnish{Työn tyyppi}
\newcommand \jyusupervisor{Supervisor Name, Other Supervisor Name}
\newcommand \jyusupervisorfinnish{Ohjaajan Nimi, Toisen Ohjaajan Nimi}
\newcommand \jyuinstitute{University of Jyväskylä}
\newcommand \jyuinstitutefinnish{Jyväskylän yliopisto}
\newcommand \jyudepartment{Department of Physics}
\newcommand \jyudepartmentfinnish{Fysiikan laitos}
\newcommand \jyuyear{2017}
\newcommand \jyumonth{05}
\newcommand \jyuday{01}

\makesetup

% These commands are optional.
\newcommand \yesnumber{\addtocounter{equation} 1\tag \theequation}
\newcommand \full{{\mathrm d}}

\begin{document}

\maketitle

\section*{Tiivistelmä}
\addcontentsline{toc}{section}{Tiivistelmä}

\noindent
\jyuauthorfinnish \\
\jyutitlefinnish \\
\jyupublicationfinnish \\
\jyudepartmentfinnish, \jyuinstitutefinnish, \jyuyear,
\pageref{p/lastpage}~sivua

\bigskip

\noindent
Tämän asiakirjaluokan tarkoitus on tarjota pohja
Jyväskylän yliopiston opiskelijoiden ja henkilökunnan julkaisuille.
Luokka esimerkkeineen on julkaistu lisenssillä GNU GPL, mikä tarkoittaa sitä,
että sillä luodut asiakirjat tulee julkaista lähdekoodeineen.

\bigskip

\noindent Avainsanat: asiakirjaluokka, latex

\section*{Abstract}
\addcontentsline{toc}{section}{Abstract}

\noindent
\jyuauthor \\
\jyutitle \\
\jyupublication \\
\jyudepartment, \jyuinstitute, \jyuyear,
\pageref{p/lastpage}~pages

\bigskip

\begin{otherlanguage}{english}
  \noindent
  This document class is intended to provide a basis
  for the publications of students and staff of the University of Jyväskylä.
  The class and its examples are released under the GNU GPL, which means that
  documents built using the class must be accompanied by their source code.
\end{otherlanguage}

\bigskip

\noindent Keywords: document class, latex

% This section is optional.
\section*{Esipuhe}
\addcontentsline{toc}{section}{Esipuhe}

Tähän voi kirjoittaa esipuheen.

\bigskip

Jyväskylässä \formatdate \jyuday \jyumonth \jyuyear

\bigskip

\jyuauthor

\tableofcontents

\section{Johdanto}
\label{s/introduction}

Tälle asiakirjaluokalle \cite{kiiskinen-2016} voi välittää
taulukon \ref{t/table} mukaisia asetuksia.

\begin{table}
  \centering
  \caption{Asiakirjaluokan asetuksia.}
  \label{t/table}
  \begin{tabular}{ll}
    \toprule
    Avainsana & Merkitys \\
    \midrule
    \texttt{finnished} & Asiakirja on kirjoitettu suomeksi \\
    \texttt{monochromatic} & Asiakirjassa on vain harmaasävykuvia \\
    \bottomrule
  \end{tabular}
\end{table}

% This section is optional.
\section{Asiakirjaluokan toteutus}
Tämän asiakirjaluokan käyttämän
ladontajärjestelmän logo on kuviossa \ref{f/figure}.

\begin{figure}
  \centering
  \rmfamily \huge \LaTeX
  \caption{Ladontajärjestelmän logo.}
  \label{f/figure}
\end{figure}

\section{Päätäntö}
\label{s/conclusions}

Tätä asiakirjaluokkaa voi käyttää vaikka
kvanttimekaniikkaan liittyvän \cite{feynman-1948} yhtälön
\begin{align*}
  \phi(R', R'') & = \int \full x \chi(t, x)^* \psi(t, x)
  \yesnumber \label{e/equation}
\end{align*}
kirjoittamiseen.

% This command is optional.
\nocite{*}

\printbibliography[heading=bibintoc]

\label{p/lastpage}

\appendix

\appendixsection{Ensimmäinen liite}
\label{s/first-attachment}

\appendixsection{Toinen liite}
\label{s/second-attachment}

\end{document}
