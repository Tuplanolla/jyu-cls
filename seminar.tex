\documentclass[final, finnished, monochromatic]{jyuseminar}

\addbibresource{thesis.bib}

\newcommand \jyuauthor{Author Name}
\newcommand \jyuauthorfinnish{Tekijän Nimi}
\newcommand \jyutitle{Title of Work}
\newcommand \jyutitlefinnish{Työn otsikko}
\newcommand \jyupublication{Type of Work}
\newcommand \jyupublicationfinnish{Työn tyyppi}
\newcommand \jyusupervisor{Supervisor Name, Other Supervisor Name}
\newcommand \jyusupervisorfinnish{Ohjaajan Nimi, Toisen Ohjaajan Nimi}
\newcommand \jyuinstitute{University of Jyväskylä}
\newcommand \jyuinstitutefinnish{Jyväskylän yliopisto}
\newcommand \jyudepartment{Department of Physics}
\newcommand \jyudepartmentfinnish{Fysiikan laitos}
\newcommand \jyuyear{2017}
\newcommand \jyumonth{05}
\newcommand \jyuday{01}

\makesetup

\beamertemplatenavigationsymbolsempty

% These commands are optional.
\newcommand \yesnumber{\addtocounter{equation} 1\tag \theequation}
\newcommand \full{{\mathrm d}}

\begin{document}

  \begin{frame}
    \titlepage
  \end{frame}

  % All of these frames are optional.

  \begin{frame}
    \frametitle{Johdanto}
    \begin{block}{Huomautus}
      Tälle asiakirjaluokalle \footfullcite{kiiskinen-2016} voi välittää
      oheisen taulukon mukaisia asetuksia.
    \end{block}

    \begin{table}
      \centering
      \caption{Asiakirjaluokan asetuksia.}
      \label{t/table}
      \begin{tabular}{ll}
        Avainsana & Merkitys \\
        \texttt{finnished} & Asiakirja on kirjoitettu suomeksi \\
        \texttt{monochromatic} & Asiakirjassa on vain harmaasävykuvia \\
      \end{tabular}
    \end{table}
  \end{frame}

  \begin{frame}
    \frametitle{Asiakirjaluokan toteutus}
    \begin{block}{Huomautus}
      Tämän asiakirjaluokan käyttämän
      ladontajärjestelmän logo on oheisessa kuviossa.
    \end{block}

    \begin{figure}
      \centering
      \rmfamily \huge \LaTeX
      \caption{Ladontajärjestelmän logo.}
    \end{figure}
  \end{frame}

  \begin{frame}
    \frametitle{Päätäntö}
    \begin{block}{Esimerkki}
      Tätä asiakirjaluokkaa voi käyttää vaikka
      kvanttimekaniikkaan liittyvän \footfullcite{feynman-1948} yhtälön
      \begin{align*}
        \phi(R', R'') & = \int_0^\infty \full x \chi(t, x)^* \psi(t, x)
        \yesnumber \label{e/equation}
      \end{align*}
      kirjoittamiseen.
    \end{block}
  \end{frame}

  % This command is optional.
  \nocite{*}

  \begin{frame}
    \frametitle{Lähteet}
    \printbibliography[heading=bibintoc]
  \end{frame}

  \appendix

  \begin{frame}
    \frametitle{Ensimmäinen liite}
  \end{frame}

  \begin{frame}
    \frametitle{Toinen liite}
  \end{frame}

\end{document}
